% !TeX spellcheck = en_US
\documentclass[12pt]{article}


\usepackage[english]{babel}
\usepackage[utf8]{inputenc}
\usepackage{graphicx}
\usepackage{amsmath}
\usepackage{mathtools}
\usepackage{hyperref}
\hypersetup{
	colorlinks   = true,    % Colours links instead of ugly boxes
	urlcolor     = blue,    % Colour for external hyperlinks
	linkcolor    = blue,    % Colour of internal links
	citecolor    = red      % Colour of citations
}
\DeclarePairedDelimiter{\ceil}{\lceil}{\rceil}



%opening
\title{Predicting Traffic Patterns in\\ Software Defined Networks}

\author{
	Liva Giovanni - liva.giovanni@spes.uniud.it
	\and
	Hermann Hellwagner - @aau.at
	\and
	Marino Miculan - @uniud.it
}

\date{\today} 

\begin{document}
	
\maketitle

\begin{abstract}
	Just the prediction part
\end{abstract}

\newpage

\section{Introduction}
The predictability of network traffic is the main aim of the thesis. 
Usually, there are two different category of network prediction: short and long period predictions.
The short forecast is used to guess values in terms of seconds or minutes. 
Instead, the long one is adopted to estimate the future workload. 
Therefore it favors the possibility to produce better planning and decision.\\
To be able to predict future load of a network, we have to create a model of its behaviors. 
On the changing of a model we have different characteristic such as the correctness of the prediction and its adaptability.


There are two type of models: \textbf{Supervised} and \textbf{Unsupervised}.
\textit{Talk about these two a little bit}

We focus over the long term prediction and supervised classifier.
The decision of the classifier have been taken conducting an experiment in a small simulated network. 
We simulate a normal scenario of daily network usage through a network of 4 nodes. 
We repeat the simulation thirty times collecting at each execution statistics of the links utilization in terms of network bandwidth and the load of the switches.
From this information, we have create different dataset changing the variables (features?) used and the numbers of lags.

.\\



\textit{Meaning of lags}

%% Parlare nella configurazione o già ora di come abbiamo fatto variare i dataset e il test su più algo differenti 




We divided the work in four phases
\begin{itemize}
	\item Observer
	\item Analyzing
	\item Plan
	\item Execute
\end{itemize}
A graphical representation is given in Figure xyz.\\
The first module, \textit{Observer}, is implemented as a daemon in the cloud application. 
Every few minute it launches a python script which queries the network controller and then stores the result inside the database.\\
The information saved regards the network load, switches and flows.
The second one, \textit{Analyzing}, is done looking through the information inside the database. 
A java application collects the data from the database and converts the knowledge in the Attribute-Relation File Format (ARFF). 
This format is an ASCII text file that describes a list of instances sharing a set of attributes.\\
The application depends on the $weka$ (Waikato Environment for Knowledge Analysis) package, a well known suite of learning machine algorithms developed by the University of Waikato.
The ARFF files are read by the weka package and used to produce the model that is adopted to make predictions.\\
The third element, \textit{Plan}, is demanded to the Administrator.
He or she can write rules to specify what to do when a particular event occurs.
In the cloud application the administrator has the possibility to create the rules that are stored inside the FloodLight controller.
The last phase, \textit{Execute}, is implemented by the controller. 
It monitors the network and every few minutes it makes predictions using the previously generated model.
When it perceives from a forecast that some rules can be applied, it fires them.\\
Every module is uncoupled from the others. This design decision of modularity gives us the possibility to change or upgrade every module whenever there is the necessity.\\



This feature is crucial for the prediction phase. We can test new classifiers or the addiction of new features in a separate and controlled network without affecting the production one. Then, we can hot swapping the model using the cloud interface.
We have designed the controller to work with a model for each switch. This decision brings the possibility to predict when a particular node will be overloaded with more precision and recall.\\

 






%%%%%%%%%%%%%%%%%%%%%%%%%%%%%%%%%%%%%%%%%%%%%%%%%%%%%%%%%%%%%%%%%%%
Net -> Mininet\\
Observer -> Daemon
Analyzer -> Weka -> Struttura modulare -> Cambiare algo quando si vuole
Plan -> Operator w/ Web Interface config the FloodLight Module
Execute -> FloodLight Module exe the rules
\section{Weka}
\subsection{Configuration}
\subsection{DataSet}
\subsection{Evaluation}
\subsection{Result}
\subsection{Discussion}
Avere 50\% di successo -> 10 volte meglio di sparare a caso $1/21 = 4.76\%$
	
	
\end{document}