% !TeX spellcheck = en_US
\documentclass[12pt]{article}


\usepackage[english]{babel}
\usepackage[utf8]{inputenc}
\usepackage{graphicx}
\usepackage{amsmath}
\usepackage{mathtools}
\usepackage{hyperref}
\hypersetup{
	colorlinks   = true,    % Colours links instead of ugly boxes
	urlcolor     = blue,    % Colour for external hyperlinks
	linkcolor    = blue,    % Colour of internal links
	citecolor    = red      % Colour of citations
}
\DeclarePairedDelimiter{\ceil}{\lceil}{\rceil}



%opening
\title{Predicting Traffic Patterns in\\ Software Defined Networks}

\author{
	Liva Giovanni - liva.giovanni@spes.uniud.it
	\and
	Marino Miculan - @uniud.it
	\and
	Hermann Hellwagner - @aau.at
}

\date{} 

\begin{document}
	
\maketitle

\begin{abstract}
	Just the prediction part
\end{abstract}


\section{Introduction}

We divided the work in four phases
\begin{itemize}
	\item Observer
	\item Analyzing
	\item Plan
	\item Execute
\end{itemize}
A graphical representation is given in Figure xyz.\\
The first module, \textit{Observer}, is implemented as a daemon in the cloud application. 
Every few minute it launches a python script which queries the network controller and then stores the result inside the database.\\
The information saved regards the network load, switches and flows.
The second one, \textit{Analyzing}, is done looking through the information inside the database. 
A java application collects the data from the database and converts the knowledge in the Attribute-Relation File Format (ARFF). 
This format is an ASCII text file that describes a list of instances sharing a set of attributes.\\
The application depends on the $weka$ (Waikato Environment for Knowledge Analysis) package, a well known suite of learning machine algorithms developed by the University of Waikato.
The ARFF files are read by the weka package and used to produce the model that is adopted to make predictions.\\
The third element, \textit{Plan}, is demanded to the Administrator.
He or she can write rules to specify what to do when a particular event occurs.
In the cloud application the administrator has the possibility to create the rules that are stored inside the FloodLight controller.
The last phase, \textit{Execute}, is implemented by the controller. 
It monitors the network and every few minutes it makes predictions using the previously generated model.
When it perceives from a forecast that some rules can be applied, it fires them.\\
Every module is uncoupled from the others. This design decision of modularity gives us the possibility to change or upgrade every module whenever there is the necessity.\\
This feature is crucial for the prediction phase. 

 






%%%%%%%%%%%%%%%%%%%%%%%%%%%%%%%%%%%%%%%%%%%%%%%%%%%%%%%%%%%%%%%%%%%
Net -> Mininet\\
Observer -> Daemon
Analyzer -> Weka -> Struttura modulare -> Cambiare algo quando si vuole
Plan -> Operator w/ Web Interface config the FloodLight Module
Execute -> FloodLight Module exe the rules
\section{Weka}
\subsection{Configuration}
\subsection{DataSet}
\subsection{Evaluation}
\subsection{Result}
\subsection{Discussion}
Avere 50\% di successo -> 10 volte meglio di sparare a caso $1/21 = 4.76\%$
	
	
\end{document}